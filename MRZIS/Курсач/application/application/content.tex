\newcommand{\applicationContent}{


    Код программы  представлен в виде блокнота Kaggle. Ссылка представлена на рисунке 1. Сcылки для просмотра датасета и быстрого скачивания для работы с датасетом в блокноте представлены соответственно на рисунках 2 и 3.
    \begin{figure}[H]
    \centering
    \includegraphics[width=0.6\textwidth]{application/qrcode_kaggle.png} 
    \caption{QR-code блокнота Kaggle}

    \begin{figure}[H]
        \centering
        \begin{minipage}{0.45\textwidth}
            \centering
            \includegraphics[width=\textwidth]{application/qrcode_dataset1.png}
            \caption{QR-code датасета для просмотра}
            \label{fig:login_page}
        \end{minipage}
        \hfill
        \begin{minipage}{0.45\textwidth}
            \centering
            \includegraphics[width=\textwidth]{application/qrcode_dataset2.png}
            \caption{QR-code датасета для быстрого скачивания для работы с датасетом в блокноте}
            \label{fig:enter_data}
        \end{minipage}
    \end{figure}
\end{figure}

   
}